% You insert your abstract in the space below.


Complex networks are structurally non-trivial and require a large set of
tools to analyze their characteristics. To understand network development,
 we introduce a set of motifs to categorize local network structure. 
Examining the Barabási–Albert and Thij models and their respective mechanisms of 
network growth, we can study temporal motif interaction. We examine how the addition of
nodes and edges affect motif composition. This analysis is insightful, but not sufficient to characterize
network dynamics.
We apply statistical correlation and covariance analysis to the dynamic motif counts to study those motifs which
move together. Still we conduct further analysis, looking to charaterize the underlying system
that drives motif counts. To this end, we use the Dynamic Mode Decomposition algorithm and the Kernel Dynamic Mode
Decomposition algorithm to find modes (spatiotemporal coherent structures) of the system. These modes offer
a dynamical systems perspective to understand the interactions between motif counts. Finally, we analyze
the results from our statistical analysis, DMD, and KDMD to understand motif interaction.